% Options for packages loaded elsewhere
\PassOptionsToPackage{unicode}{hyperref}
\PassOptionsToPackage{hyphens}{url}
\PassOptionsToPackage{dvipsnames,svgnames,x11names}{xcolor}
%
\documentclass[
  letterpaper,
  DIV=11,
  numbers=noendperiod]{scrartcl}

\usepackage{amsmath,amssymb}
\usepackage{iftex}
\ifPDFTeX
  \usepackage[T1]{fontenc}
  \usepackage[utf8]{inputenc}
  \usepackage{textcomp} % provide euro and other symbols
\else % if luatex or xetex
  \usepackage{unicode-math}
  \defaultfontfeatures{Scale=MatchLowercase}
  \defaultfontfeatures[\rmfamily]{Ligatures=TeX,Scale=1}
\fi
\usepackage{lmodern}
\ifPDFTeX\else
    % xetex/luatex font selection
\fi
% Use upquote if available, for straight quotes in verbatim environments
\IfFileExists{upquote.sty}{\usepackage{upquote}}{}
\IfFileExists{microtype.sty}{% use microtype if available
  \usepackage[]{microtype}
  \UseMicrotypeSet[protrusion]{basicmath} % disable protrusion for tt fonts
}{}
\makeatletter
\@ifundefined{KOMAClassName}{% if non-KOMA class
  \IfFileExists{parskip.sty}{%
    \usepackage{parskip}
  }{% else
    \setlength{\parindent}{0pt}
    \setlength{\parskip}{6pt plus 2pt minus 1pt}}
}{% if KOMA class
  \KOMAoptions{parskip=half}}
\makeatother
\usepackage{xcolor}
\setlength{\emergencystretch}{3em} % prevent overfull lines
\setcounter{secnumdepth}{-\maxdimen} % remove section numbering
% Make \paragraph and \subparagraph free-standing
\makeatletter
\ifx\paragraph\undefined\else
  \let\oldparagraph\paragraph
  \renewcommand{\paragraph}{
    \@ifstar
      \xxxParagraphStar
      \xxxParagraphNoStar
  }
  \newcommand{\xxxParagraphStar}[1]{\oldparagraph*{#1}\mbox{}}
  \newcommand{\xxxParagraphNoStar}[1]{\oldparagraph{#1}\mbox{}}
\fi
\ifx\subparagraph\undefined\else
  \let\oldsubparagraph\subparagraph
  \renewcommand{\subparagraph}{
    \@ifstar
      \xxxSubParagraphStar
      \xxxSubParagraphNoStar
  }
  \newcommand{\xxxSubParagraphStar}[1]{\oldsubparagraph*{#1}\mbox{}}
  \newcommand{\xxxSubParagraphNoStar}[1]{\oldsubparagraph{#1}\mbox{}}
\fi
\makeatother


\providecommand{\tightlist}{%
  \setlength{\itemsep}{0pt}\setlength{\parskip}{0pt}}\usepackage{longtable,booktabs,array}
\usepackage{calc} % for calculating minipage widths
% Correct order of tables after \paragraph or \subparagraph
\usepackage{etoolbox}
\makeatletter
\patchcmd\longtable{\par}{\if@noskipsec\mbox{}\fi\par}{}{}
\makeatother
% Allow footnotes in longtable head/foot
\IfFileExists{footnotehyper.sty}{\usepackage{footnotehyper}}{\usepackage{footnote}}
\makesavenoteenv{longtable}
\usepackage{graphicx}
\makeatletter
\def\maxwidth{\ifdim\Gin@nat@width>\linewidth\linewidth\else\Gin@nat@width\fi}
\def\maxheight{\ifdim\Gin@nat@height>\textheight\textheight\else\Gin@nat@height\fi}
\makeatother
% Scale images if necessary, so that they will not overflow the page
% margins by default, and it is still possible to overwrite the defaults
% using explicit options in \includegraphics[width, height, ...]{}
\setkeys{Gin}{width=\maxwidth,height=\maxheight,keepaspectratio}
% Set default figure placement to htbp
\makeatletter
\def\fps@figure{htbp}
\makeatother

\newcommand{\vect}[1]{\mathbf{#1}}
\newcommand{\m}[1]{\mathbf{#1}}
\newcommand{\rand}[1]{\textnormal{#1}}
\newcommand{\vrand}[1]{\mathbf{#1}}
\newcommand{\tensor}[1]{\mathbf{\mathsf{#1}}}
\newcommand{\N}{\mathbb{N}}
\newcommand{\floor}[1]{\lfloor#1\rfloor}
\newcommand{\bmat}{\left[\begin{array}}
\newcommand{\emat}{\end{array}\right]}
\KOMAoption{captions}{tableheading}
\makeatletter
\@ifpackageloaded{caption}{}{\usepackage{caption}}
\AtBeginDocument{%
\ifdefined\contentsname
  \renewcommand*\contentsname{Table of contents}
\else
  \newcommand\contentsname{Table of contents}
\fi
\ifdefined\listfigurename
  \renewcommand*\listfigurename{List of Figures}
\else
  \newcommand\listfigurename{List of Figures}
\fi
\ifdefined\listtablename
  \renewcommand*\listtablename{List of Tables}
\else
  \newcommand\listtablename{List of Tables}
\fi
\ifdefined\figurename
  \renewcommand*\figurename{Figure}
\else
  \newcommand\figurename{Figure}
\fi
\ifdefined\tablename
  \renewcommand*\tablename{Table}
\else
  \newcommand\tablename{Table}
\fi
}
\@ifpackageloaded{float}{}{\usepackage{float}}
\floatstyle{ruled}
\@ifundefined{c@chapter}{\newfloat{codelisting}{h}{lop}}{\newfloat{codelisting}{h}{lop}[chapter]}
\floatname{codelisting}{Listing}
\newcommand*\listoflistings{\listof{codelisting}{List of Listings}}
\makeatother
\makeatletter
\makeatother
\makeatletter
\@ifpackageloaded{caption}{}{\usepackage{caption}}
\@ifpackageloaded{subcaption}{}{\usepackage{subcaption}}
\makeatother

\ifLuaTeX
  \usepackage{selnolig}  % disable illegal ligatures
\fi
\usepackage{bookmark}

\IfFileExists{xurl.sty}{\usepackage{xurl}}{} % add URL line breaks if available
\urlstyle{same} % disable monospaced font for URLs
\hypersetup{
  pdftitle={Helpful Links \& Resources},
  pdfauthor={Marco Willi},
  colorlinks=true,
  linkcolor={blue},
  filecolor={Maroon},
  citecolor={Blue},
  urlcolor={Blue},
  pdfcreator={LaTeX via pandoc}}


\title{Helpful Links \& Resources}
\author{Marco Willi}
\date{}

\begin{document}
\maketitle


Links and ressources to different topics related to Machine Learning,
Deep Learning, and Images.

\subsection{Theory}\label{theory}

\subsubsection{PyTorch}\label{pytorch}

\href{http://blog.ezyang.com/2019/05/pytorch-internals/}{PyTorch
internals - Blog Post}

\subsubsection{Deep Learning and Computer
Vision}\label{deep-learning-and-computer-vision}

\href{https://www.youtube.com/playlist?list=PL5-TkQAfAZFbzxjBHtzdVCWE0Zbhomg7r}{University
of Michigan - Deep Learning for Computer Vision}

\begin{itemize}
\tightlist
\item
  Sehr gute Vorlesung zum Thema
\end{itemize}

\href{https://fluff-armadillo-037.notion.site/Modern-Computer-Vision-and-Deep-Learning-CS-198-126-b11006739378470fa67a9cf6594201e0}{University
of California, Berkeley - Modern Computer Vision and Deep Learning}

\begin{itemize}
\tightlist
\item
  Sehr gute Vorlesung zum Thema
\end{itemize}

\subsubsection{Neuronale Netzwerke - Basics}\label{sec-links-nn}

\href{https://www.youtube.com/watch?v=C8Uns9HEVXI}{Perceptron Learning
Rule S. Raschka}

\href{https://cs229.stanford.edu/notes2020spring/cs229-notes-deep_learning.pdf}{CS229
Stanford MLP Backpropagation}

\href{https://www.ics.uci.edu/~pjsadows/notes.pdf}{Notes on
Backpropagation}

\href{https://youtu.be/IHZwWFHWa-w}{3Blue1Brown Gradient Descent}

\href{https://www.youtube.com/watch?v=tIeHLnjs5U8}{3Blue1Brown
Backpropagation Calculus}

\href{https://youtu.be/mO7BpWmzT78}{Andrew Ng Backprop}

\href{https://www.youtube.com/watch?v=VMj-3S1tku0}{Andrej Karpathy -
Backpropagation from the ground up}

\subsubsection{Model Selection}\label{model-selection}

\href{https://arxiv.org/pdf/1811.12808.pdf}{Paper von S.Raschka: ``Model
Evaluation, Model Selection, and Algorithm Selection in Machine
Learning''}

\subsection{Practical}\label{practical}

\href{http://karpathy.github.io/2019/04/25/recipe/}{Andrej Karpathy - A
Recipe for Training Neural Networks}

\subsubsection{ML Best Practices Videos}\label{ml-best-practices-videos}

\href{https://developers.google.com/machine-learning/guides/rules-of-ml}{Martin
Zinkevich - Best Practices for ML Engineering}

\href{https://www.youtube.com/watch?v=sZSKGNbrwus&list=PLLssT5z_DsK-h9vYZkQkYNWcItqhlRJLN&index=59}{Andrew
Ng - Advice For Applying Machine Learning \textbar{} Deciding What To
Try Next}

\href{https://www.youtube.com/watch?v=ISBGFY-gBug&list=PLLssT5z_DsK-h9vYZkQkYNWcItqhlRJLN&index=64}{Andrew
Ng - Advice For Applying Machine Learning \textbar{} Learning Curves}

\href{https://www.youtube.com/watch?v=yoYA1MFpYRg&list=PLLssT5z_DsK-h9vYZkQkYNWcItqhlRJLN&index=64}{Andrew
Ng - Advice For Applying Machine Learning \textbar{} Deciding What To Do
Next (Revisited)}

\href{https://www.youtube.com/watch?v=HREeLryOh4Q&list=PLLssT5z_DsK-h9vYZkQkYNWcItqhlRJLN&index=65}{Andrew
Ng - Machine Learning System Design \textbar{} Prioritizing What To Work
On}

\href{https://www.youtube.com/watch?v=k1JGvqr56Yk&list=PLLssT5z_DsK-h9vYZkQkYNWcItqhlRJLN&index=66}{Andrew
Ng - Machine Learning System Design \textbar{} Error Analysis}

\href{https://www.youtube.com/watch?v=5T77nG7YJhk&list=PLLssT5z_DsK-h9vYZkQkYNWcItqhlRJLN&index=69}{Andrew
Ng - Machine Learning System Design \textbar{} Data For Machine
Learning}

\subsection{Tools}\label{tools}

\subsubsection{Data Science Repository}\label{data-science-repository}

\href{https://khuyentran1401.github.io/reproducible-data-science/structure_project/introduction.html}{Build
a Reproducible and Maintainable Data Science Project}

\begin{itemize}
\tightlist
\item
  great jupyter book to learen about how to structure a repository and
  more
\end{itemize}

\href{https://github.com/ashleve/lightning-hydra-template}{Lightning-Hydra-Template}

\begin{itemize}
\tightlist
\item
  template to strcuture a repository based on experiment configuration
  with Hydra and Pytorch-Lightning
\end{itemize}

\subsubsection{Data Handling}\label{data-handling}

\href{https://huggingface.co/docs/datasets/en/index}{datasets}

\begin{itemize}
\tightlist
\item
  Great package to create and manage (large) image datasets
\end{itemize}

\href{https://github.com/rom1504/img2dataset}{img2dataset}

\begin{itemize}
\tightlist
\item
  Package to download large image datasets from urls
\end{itemize}

\href{https://dvc.org/}{DVC}

\begin{itemize}
\tightlist
\item
  Package for data version control
\end{itemize}

\subsubsection{PyTorch}\label{pytorch-1}

\href{https://lightning.ai/docs/pytorch/stable/}{Lightning}

\begin{itemize}
\tightlist
\item
  boilerplate code to easily train models and use gpu, etc.
\end{itemize}




\end{document}
